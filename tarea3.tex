% Tipo de documento 
\documentclass{article}

% Idioma
\usepackage[spanish]{babel}

% Márgenes amplios
\usepackage[margin = 1.5cm]{geometry}

% Estructuras lógicas
\usepackage{bussproofs}

% Algunos comandos pra trabajar con ecuaciones (alinear)
\usepackage{amsmath}

\begin{document}
    % Título
    \title{Autómatas y Lenguajes formales \\
    \large 2912-2 \\
    \large Ejercicio semanal 3}

    \date{Fecha de entrega: 15 de febrero del 2019}

    \author{Sandra del Mar Soto Corderi \\
            Edgar Quiroz Castañeda}  
    \maketitle
    
    % Ejercicios
    \begin{enumerate}
        \item {
            Sea $L \subset \{a, b\}^*$ definido como 
            \begin{align*}
                \AxiomC{}
                \UnaryInfC{$\epsilon \in L$}
                \DisplayProof \\
                \\
                \AxiomC{$x \in L $}
                \UnaryInfC{$xa \in L$}
                \DisplayProof \\
                \\
                \AxiomC{$x \in L $}
                \UnaryInfC{$xba \in L$}
                \DisplayProof \\
            \end{align*}

            Demuestre que para toda $x \in L$ se cumple que
            \begin{enumerate}
                \item {
                    $n_a(x) \geq n_b(x)$.
                }
                \item {
                    $x$ no contiene la subcadena $bb$.
                }
            \end{enumerate}
            
        }
        \item {
            Considere las expresiones regulares
            \begin{align*}
                r &= a^* + b^* \\
                s &= ab^* + ba^* + b^*a + (a^*b)^*
            \end{align*}
            
            Da una cadena que cumpla lo siguiente o justifica porque no existe.
            \begin{enumerate}
                \item {
                    Que corresponda a $r$ pero no a $s$.
                }
                \item {
                    Que corresponda a $s$ pero no a $r$.
                }
                \item {
                    Que corresponda tanto a $r$ como a $s$.
                }
            \end{enumerate}
        }
    \end{enumerate}
\end{document}