% Tipo de documento 
\documentclass{article}

% Idioma
\usepackage[spanish]{babel}

% Márgenes amplios
\usepackage[margin = 1.5cm]{geometry}

% Estructuras lógicas
\usepackage{bussproofs}

% Algunos comandos para trabajar con ecuaciones (alinear)
\usepackage{amsmath}

%Paquete para español y caracteres especiales
\usepackage[spanish]{babel}
\usepackage[utf8]{inputenc}

\begin{document}
    % Título
    \title{Autómatas y Lenguajes formales \\
    \large 2019-2 \\
    \large Ejercicio semanal 3}

    \date{Fecha de entrega: 15 de febrero del 2019}

    \author{Sandra del Mar Soto Corderi \\
            Edgar Quiroz Castañeda}  
    \maketitle
    
    % Ejercicios
    \begin{enumerate}
        \item {
            Sea $L \subset \{a, b\}^*$ definido como 
            \begin{align*}
                \AxiomC{}
                \UnaryInfC{$\epsilon \in L$}
                \DisplayProof \\
                \\
                \AxiomC{$x \in L $}
                \UnaryInfC{$xa \in L$}
                \DisplayProof \\
                \\
                \AxiomC{$x \in L $}
                \UnaryInfC{$xba \in L$}
                \DisplayProof \\
            \end{align*}

            Demuestre que para toda $x \in L$ se cumple que
            \begin{enumerate}
                \item {
                    $n_a(x) \geq n_b(x)$.\\
                    Como $L$ tiene una definición recursiva, tiene un principio 
                    de inducción estructural.\\
                    Entonces, demostremos esta propiedad por inducción.
                    \begin{itemize}
                        \item {
                            Caso base $x = \epsilon$ \\
                            Entonces, $n_a(x) = n_a(\epsilon) = 0 = 
                            n_b(\epsilon) = n_b(x)$, por lo que en 
                            particular, $n_a(x) \geq n_b(x)$
                        }
                        \item {
                            Hipótesis \\
                            Sea $x \in L$. Entonces, $n_a(x) \geq n_b(x)$.
                        }
                        \item {
                            Paso inductivo.\\
                            Sea $w \in L$ construida usando una de las reglas 
                            recursivas de la definición de $L$.\\
                            Como hay dos reglas recursivas, entonces hay dos 
                            casos para $w$.
                            \begin{itemize}
                                \item {
                                    Caso 1, $w = xa$.\\
                                    Entonces, $n_a(xa) = n_a(x) + 1$ y
                                    $n_b(xa) = n_b(x)$.\\ 
                                    Y por hipótesis, tenemos que 
                                    $n_a(x) \geq n_b(x)$, por lo que que 
                                    $n_a(x) + 1 > n_b(x)$.\\
                                    Y como es mayor, en particular 
                                    $n_a(xa) = n_a(x) + 1 \geq n_b(x) = n_b(xa)$
                                }
                                \item {
                                    Caso 2, $w = xba$.\\
                                    Entonces, $n_a(xba) = n_a(x) + 1$ y
                                    $n_b(xba) = n_b(x) + 1$.\\ 
                                    Y por hipótesis, tenemos que 
                                    $n_a(x) \geq n_b(x)$, por lo que que 
                                    $n_a(xba) = n_a(x) + 1 \geq n_b(x) + 1 
                                    = n_b(xba)$.
                                }
                            \end{itemize}
                        }
                    \end{itemize}
                }
                \item {
                    $x$ no contiene la subcadena $bb$.\\
                    Como $L$ tiene una definición recursiva, tiene un principio 
                    de inducción estructural.\\
                    Pero antes de demostrarla, demostremos una propiedad auxiliar:
                    $x$ no termina en $b$.\\
                    Hay una regla básica y dos recursivas. Veamos que esto se 
                    cumple para esas tres reglas.

                    \begin{itemize}
                        \item {
                            Para la regla base, con $x = \epsilon$ tenemos que 
                            $\epsilon$ no tiene ningún símbolo, por lo que en 
                            particular no termina en $b$.
                        }
                        \item {
                            Para ambas reglas recursivas, tenemos que 
                            $x = wa$ o $w = wba$ para alguna $w \in L$. Como 
                            en ambos casos termina en $a$, no termina en $b$.
                        }
                    \end{itemize}
                    Por lo que ninguna cadena de $L$ termina en $b$. 

                    Volviendo al problema principal, demostremos la propiedad 
                    por inducción.
                    \begin{itemize}
                        \item {
                            Case base $x = \epsilon$.\\
                            Como $\epsilon$ no tiene ningún símbolo, entonces en
                            particular no tiene ninguna $bb$.
                        }
                        \item {
                            Hipótesis
                            Sea $x \in L$. Entonces, $x$ no tiene como subcadena 
                            $bb$.
                        }
                        \item {
                            Paso inductivo.
                            Sea $w \in L$ construida usando una de las reglas 
                            recursivas de la definición de $L$.\\
                            Como hay dos reglas recursivas, entonces hay dos 
                            casos para $w$.
                            \begin{itemize}
                                \item {
                                    Caso 1, $w = xa$.\\
                                    Por hipótesis, $x$ no tiene como subcadena a
                                    $bb$. Y como a $xa$ sólo se le concatena 
                                    una $a$, entonces $xa$ tampoco puede tener
                                    como subcadena a $bb$.
                                }
                                \item {
                                    Case 2, $w = xba$. \\
                                    Por hipótesis, $x$ no tiene como subcadena a
                                    $bb$. Y como a $xba$ sólo se le concatena 
                                    una $ba$, y $x$ no termina en $b$ por lo
                                    demostrado anteriormente,  $xba$ tampoco 
                                    puede tener como subcadena a $bb$.
                                }
                            \end{itemize}
                        }
                    \end{itemize}
                }
            \end{enumerate}
            
        }
        \item {
            Considere las expresiones regulares
            \begin{align*}
                r &= a^* + b^* \\
                s &= ab^* + ba^* + b^*a + (a^*b)^*
            \end{align*}
            
            Da una cadena que cumpla lo siguiente o justifica porque no existe.
            \begin{enumerate}
                \item {
                    Que corresponda a $r$ pero no a $s$.
                    
                    Proponemos la cadena w = $\bf{aaaaaa}$\\
                }
                \item {
                    Que corresponda a $s$ pero no a $r$.
                    
                   Proponemos la cadena w = $\bf{abbb}$\\
                   
                }
                \item {
                    Que corresponda tanto a $r$ como a $s$.
                    
                    Proponemos la cadena w = $\bf{\epsilon}$
                }
            \end{enumerate}
        }
    \end{enumerate}
\end{document}